\documentclass[a4paper,12pt]{article}
\usepackage[utf8]{inputenc}
\usepackage[MeX]{polski}
\usepackage{fullpage}

\title{SPOP 2014 L. Projekt. Opis źródeł.}
\author{Aniserowicz Michał, Turek Jakub}
\date{}

\begin{document}

\maketitle

\section{Moduły}

Program składa się z następujących modułów:

\begin{itemize}
 \item \verb+Logic.Game+ - zawiera typy i funkcje realizujące zasady gry ``Wilk i Owce'', w tym heurystyczną funkcję oceny stanu gry z punktu widzenia danego gracza;
 \item \verb+Logic.AI+ - zawiera funkcje realizujące sztuczną inteligencję, w tym implementację algorytmu minimax wykorzystującego cięcia alfa-beta;
 \item \verb+Presentation.GameLoop+ - zawiera funkcje implementujące pętlę gry: wyświetlenie i obsługę opcji, wykonanie ruchu komputera itp.;
 \item \verb+Presentation.GameSpecific+ - zawiera funkcje specyficzne dla gry ``Wilk i Owce'': inicjalizację planszy (wybór pozycji Wilka), wykonanie ruchu gracza itp.;
 \item \verb+Presentation.SaveLoad+ - zawiera funkcje realizujące odczyt/zapis gry z/do pliku;
 \item \verb+Main+ - główny moduł programu, inicjalizuje pętlę gry.
\end{itemize}


\section{Heurystyka}

Heurystyczna funkcja oceny stanu gry to znormalizowana suma ``punktów'' przyznawanych za następujące aspekty stanu planszy:
\begin{itemize}
 \item liczba możliwych ruchów Wilka: $0~\rightarrow~12$~pkt., $1~\rightarrow~9$~pkt., $2~\rightarrow~6$~pkt. itd.;
 \item odległość Wilka od "owczej`` krawędzi planszy (im większa tym lepiej): $0~-~7$ pkt.;
 \item minimalna odległóść Owcy od ''wilczej`` krawędzi planszy (im większa tym lepiej): $0~-~7$ pkt.
\end{itemize}

Suma ta przyjmuje wartości z przedziału $[0; 26]$.
Normalizacja polega na pomniejszeniu jej wyniku o $13$~pkt., aby uzyskać wartość z przedziału $[-13; 13]$.
Stąd, ocena zwycięstwa to $14$~pkt., a porażki - $-14$~pkt.

Głębokość przeszukiwania drzewa gry została ustalona eksperymentalnie na $11$.
Wartość tę można zmienić, modyfikując stałą \verb+Logic.AI.defaultAlphaBetaHeuristicDepth+.


\end{document}